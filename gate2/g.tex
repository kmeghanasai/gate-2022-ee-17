\let\negmedspace\undefined
\let\negthickspace\undefined
\documentclass[journal,12pt,onecolumn]{IEEEtran}
\usepackage{cite}
\usepackage{amsmath,amssymb,amsfonts,amsthm}
\usepackage{algorithmic}
\usepackage{graphicx}
\usepackage{textcomp}
\usepackage{tikz}
\usepackage{xcolor}
\usepackage{txfonts}
\usepackage{listings}
\usepackage{pgfplots}
\usepackage{enumitem}
\usepackage{mathtools}
\usepackage{gensymb}

\usepackage{tkz-euclide} % loads  TikZ and tkz-base
\usepackage{listings}



\newtheorem{theorem}{Theorem}[section]
\newtheorem{problem}{Problem}
\newtheorem{proposition}{Proposition}[section]
\newtheorem{lemma}{Lemma}[section]
\newtheorem{corollary}[theorem]{Corollary}
\newtheorem{example}{Example}[section]
\newtheorem{definition}[problem]{Definition}
%\newtheorem{thm}{Theorem}[section] 
%\newtheorem{defn}[thm]{Definition}
%\newtheorem{algorithm}{Algorithm}[section]
%\newtheorem{cor}{Corollary}
\newcommand{\BEQA}{\begin{eqnarray}}
\newcommand{\EEQA}{\end{eqnarray}}
\newcommand{\system}[1]{\stackrel{#1}{\rightarrow}}

\newcommand{\define}{\stackrel{\triangle}{=}}
\theoremstyle{remark}
\newtheorem{rem}{Remark}
%\bibliographystyle{ieeetr}
\begin{document}
%
\providecommand{\pr}[1]{\ensuremath{\Pr\left(#1\right)}}
\providecommand{\prt}[2]{\ensuremath{p_{#1}^{\left(#2\right)} }}        % own macro for this question
\providecommand{\qfunc}[1]{\ensuremath{Q\left(#1\right)}}
\providecommand{\sbrak}[1]{\ensuremath{{}\left[#1\right]}}
\providecommand{\lsbrak}[1]{\ensuremath{{}\left[#1\right.}}
\providecommand{\rsbrak}[1]{\ensuremath{{}\left.#1\right]}}
\providecommand{\brak}[1]{\ensuremath{\left(#1\right)}}
\providecommand{\lbrak}[1]{\ensuremath{\left(#1\right.}}
\providecommand{\rbrak}[1]{\ensuremath{\left.#1\right)}}
\providecommand{\cbrak}[1]{\ensuremath{\left\{#1\right\}}}
\providecommand{\lcbrak}[1]{\ensuremath{\left\{#1\right.}}
\providecommand{\rcbrak}[1]{\ensuremath{\left.#1\right\}}}
\newcommand{\sgn}{\mathop{\mathrm{sgn}}}
\providecommand{\abs}[1]{\left\vert#1\right\vert}
\providecommand{\res}[1]{\Res\displaylimits_{#1}} 
\providecommand{\norm}[1]{\left\lVert#1\right\rVert}
%\providecommand{\norm}[1]{\lVert#1\rVert}
\providecommand{\mtx}[1]{\mathbf{#1}}
\providecommand{\mean}[1]{E\left[ #1 \right]}
\providecommand{\cond}[2]{#1\middle|#2}
\providecommand{\fourier}{\overset{\mathcal{F}}{ \rightleftharpoons}}
\newenvironment{amatrix}[1]{%
  \left(\begin{array}{@{}*{#1}{c}|c@{}}
}{%
  \end{array}\right)
}
%\providecommand{\hilbert}{\overset{\mathcal{H}}{ \rightleftharpoons}}
%\providecommand{\system}{\overset{\mathcal{H}}{ \longleftrightarrow}}
	%\newcommand{\solution}[2]{\textbf{Solution:}{#1}}
\newcommand{\solution}{\noindent \textbf{Solution: }}
\newcommand{\cosec}{\,\text{cosec}\,}
\providecommand{\dec}[2]{\ensuremath{\overset{#1}{\underset{#2}{\gtrless}}}}
\newcommand{\myvec}[1]{\ensuremath{\begin{pmatrix}#1\end{pmatrix}}}
\newcommand{\mydet}[1]{\ensuremath{\begin{vmatrix}#1\end{vmatrix}}}
\newcommand{\myaugvec}[2]{\ensuremath{\begin{amatrix}{#1}#2\end{amatrix}}}
\providecommand{\rank}{\text{rank}}
\providecommand{\pr}[1]{\ensuremath{\Pr\left(#1\right)}}
\providecommand{\qfunc}[1]{\ensuremath{Q\left(#1\right)}}
	\newcommand*{\permcomb}[4][0mu]{{{}^{#3}\mkern#1#2_{#4}}}
\newcommand*{\perm}[1][-3mu]{\permcomb[#1]{P}}
\newcommand*{\comb}[1][-1mu]{\permcomb[#1]{C}}
\providecommand{\qfunc}[1]{\ensuremath{Q\left(#1\right)}}
\providecommand{\gauss}[2]{\mathcal{N}\ensuremath{\left(#1,#2\right)}}
\providecommand{\diff}[2]{\ensuremath{\frac{d{#1}}{d{#2}}}}
\providecommand{\myceil}[1]{\left \lceil #1 \right \rceil }
\newcommand\figref{Fig.~\ref}
\newcommand\tabref{Table~\ref}
\newcommand{\sinc}{\,\text{sinc}\,}
\newcommand{\rect}{\,\text{rect}\,}
%%
%	%\newcommand{\solution}[2]{\textbf{Solution:}{#1}}
%\newcommand{\solution}{\noindent \textbf{Solution: }}
%\newcommand{\cosec}{\,\text{cosec}\,}
%\numberwithin{equation}{section}
%\numberwithin{equation}{subsection}
%\numberwithin{problem}{section}
%\numberwithin{definition}{section}
%\makeatletter
%\@addtoreset{figure}{problem}
%\makeatother

%\let\StandardTheFigure\thefigure
\let\vec\mathbf

\bibliographystyle{IEEEtran}





\bigskip


\title{Gate 2022\_EE\_17}
\author{Karyampudi Meghana Sai\\ EE23BTECH11031}

\maketitle

The Bode magnitude plot of a first order stable system is constant with frequency. The asymptotic value of the high frequency phase, for the system, is $-180\degree$ . This system has \\
\begin{figure}[h]
    \centering
    \resizebox{0.55\columnwidth}{!}{\begin{circuitikz}
    % Draw the components and connect them
    \draw (0,2) to[L, l=$44\,mH$] (4,2)
    (0,0) to[V, v=$220\,V$, f=$50\,Hz$] (4,0)
    (0,2) -- (0,0)
    (4,2) -- (4,0);
\end{circuitikz}
}
    \caption{}
    \label{fig:gate2022ee17fig1}
\end{figure} \\
(A) one LHP pole and one RHP zero at the same frequency.\\
(B) one LHP pole and one LHP zero at the same frequency.\\
(C) two LHP poles and one RHP zero.\\
(D) two RHP poles and one LHP zero.\\
\hfill Gate 2022 EE 17

\solution\\
Flat constant magnitude response for all frequency of system shows that it is an all pass system.\\
In all pass system, poles and zeros are symmetrical about $j\omega$ axis.\\
Possible transfer functions are\\
\begin{align}
T_1 (s)&=\frac{s-a}{s+a} \quad a>0 \label{eq:gate2022eee17eq1}\\
T_2 (s)&=\frac{a-s}{a+s} \quad a>0 \label{eq:gate2022eee17eq2}\\
s&=j\omega\label{eq:gate2022eee17eq3}
\end{align}
From the phase plot as $\omega \system{} \infty$ shows $\phi= -180\degree$.\\
\begin{enumerate}
\item For $T_1(s)$:\\
Using equation \eqref{eq:gate2022eee17eq3}\\
\begin{align}
T_1 (j \omega)&=\frac{j \omega -a}{j \omega +a}\\
\angle T_1(j\omega)&=180\degree - \tan^{-1}\brak{\frac{\omega}{a}}-\tan^{-1}\brak{\frac{\omega}{a}}\\
&=180\degree - 2\tan^{-1}\brak{\frac{\omega}{a}}
\end{align}
At $\omega=\infty$,
\begin{align}
\angle T_1(j\omega)=0\degree
\end{align}
\item For $T_2(s)$:\\
Using equation \eqref{eq:gate2022eee17eq3}\\
\begin{align}
T_2 (j \omega)&=\frac{a-j \omega }{a+j \omega}\\
\angle T_2(j\omega)&=- \tan^{-1}\brak{\frac{\omega}{a}}-\tan^{-1}\brak{\frac{\omega}{a}}\\
&=- 2\tan^{-1}\brak{\frac{\omega}{a}}
\end{align}
At $\omega=\infty$,
\begin{align}
\angle T_2(j\omega)=-180\degree
\end{align}
\end{enumerate}
Hence, the transfer function of given all pass filter.\\
\begin{align}
T(s)=\frac{a-s}{a+s} \quad a>0
\end{align}
Hence, the system has one LHP pole and one RHP zero at the same frequency.
\end{document}
